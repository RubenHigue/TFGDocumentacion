% ============================================================|
%                                                             |
% Plantilla para TFG del Dpto. de Informática de la EPSJ v1.0 |
%                                                             |
% Ajustada según la normativa de estilo indicada por la EPSJ  |
% en el documento https://eps.ujaen.es/sites/centro_epsj/files|
% /uploads/documents/grados/TFG/criteriosYestilo_TFG.pdf      |
%                                                             |
%    - A4 con márgenes de 2.5cm                               |
%    - Intelineado de 1.5 líneas                              |
%    - Primera línea de cada párrafo con sangrado             |
%    - Fuente Arial de 12pt                                   |
%    - Cabecera con nombre de estudiante y título de TFG      |
%    - Pie con nombre del centro y nº de página               |
%    - Portada oficial EPSJ - Dpto. Informática con datos     |
%                                                             |
% 2020/09/25 - Francisco Charte Ojeda                         |
% CC0 1.0 Universal (CC0 1.0)                                 |
% https://creativecommons.org/publicdomain/zero/1.0/deed.es   |
%                                                             |
%=============================================================|

\usepackage[margin=2.5cm]{geometry}

\usepackage{setspace} % Para fijar el interlineado

\usepackage{fancyhdr} % Para los encabezados y pies
\pagestyle{fancy}
\fancyhf{}

\usepackage{lipsum} % Para generar texto como contenido de prueba, se puede eliminar en la memoria final

\usepackage{helvet}

\usepackage{wallpaper}   % Para introducir el PDF con la portada EPSJ
\usepackage[absolute,overlay]{textpos}

\usepackage{etoolbox}  % Herramientas y bibliografía
\usepackage[numbers]{natbib}
\usepackage{csquotes}

\usepackage{hyperref}  % Para URL como enlaces
\hypersetup{
    colorlinks=true,
    linkcolor=blue,
    filecolor=magenta,      
    urlcolor=cyan,
    pdftitle={\TituloTFG},
    pdfauthor={\Estudiante}
    bookmarks=true,
    bookmarksopen=true,
    pdfpagemode=FullScreen,
    breaklinks=true,
    citecolor=cyan,
}

\usepackage{booktabs}  % Para tablas con mejor apariencia

\usepackage[ruled,vlined]{algorithm2e} % Para incluir algoritmos
\renewcommand{\algorithmcfname}{Algoritmo}
\renewcommand{\listalgorithmcfname}{Lista de algoritmos}

\usepackage{listings}  % Para incluir listados de código
\usepackage{xcolor}
\lstset{ %
  %language=delphi,                % the language of the code
  basicstyle=\linespread{0.7}\small\ttfamily,       % the size of the fonts that are used for the code
  numbers=left,                   % where to put the line-numbers
  numberstyle=\footnotesize\color{gray},  % the style that is used for the line-numbers
  stepnumber=1,                   % the step between two line-numbers. If it's 1, each line will be numbered
  numbersep=7pt,                  % how far the line-numbers are from the code
  backgroundcolor=\color{gray!5},  % choose the background color. You must add \usepackage{color}
  showspaces=false,               % show spaces adding particular underscores
  showstringspaces=true,         % underline spaces within strings
  showtabs=true,                 % show tabs within strings adding particular underscores
  frameround=fttt,
  frame=rtBL,                   % adds a frame around the code
  rulecolor=\color{black},        % if not set, the frame-color may be changed on line-breaks within not-black text (e.g. commens (green here))
  tabsize=4,                      % sets default tabsize to 2 spaces
  aboveskip=1em,
  captionpos=b,                   % sets the caption-position to bottom
  breaklines=true,                % sets automatic line breaking
  breakatwhitespace=false,        % sets if automatic breaks should only happen at whitespace
  title=\lstname,                 % show the filename of files included with \lstinputlisting;  also try caption instead of title
  keywordstyle=\bf\ttfamily,          % keyword style
  commentstyle=\color{black!60}\ttfamily,       % comment style
  stringstyle=\color{blue}\ttfamily,         % string literal style
  escapeinside={\%*}{*)}            % if you want to add a comment within your code
}
\renewcommand{\lstlistingname}{Listado}
\renewcommand{\lstlistlistingname}{Lista de listados de código}
\lstset
{
    language=[LaTeX]TeX,
    breaklines=true,
    basicstyle=\tt\scriptsize,
    keywordstyle=\color{blue},
    identifierstyle=\color{gray},
    texcl=true
}
\lstset{
    language=Python,
    literate={á}{{\'a}}1
        {ã}{{\~a}}1
        {é}{{\'e}}1
        {ó}{{\'o}}1
        {í}{{\'i}}1
        {ñ}{{\~n}}1
        {¡}{{!`}}1
        {¿}{{?`}}1
        {ú}{{\'u}}1
        {Í}{{\'I}}1
        {Ó}{{\'O}}1
}

\renewcommand{\familydefault}{\sfdefault}
\setlength{\parskip}{1em}

% Configuración de encabezado y pie
\fancyhead[RE,LO]{{\color{gray}\Estudiante}}
\fancyhead[LE,RO]{{\color{gray}\TituloAbreviado}}
\fancyfoot[RE,LO]{{\color{gray}Escuela Politécnica Superior de Jaén}}
\fancyfoot[LE,RO]{{\color{gray}\thepage}}
\renewcommand{\footrulewidth}{1pt}

\renewcommand{\spanishtablename}{Tabla.}  % Título para las tablas
\renewcommand{\spanishcontentsname}{Tabla de contenidos}  % y los índices
\renewcommand{\spanishlistfigurename}{Lista de figuras}
\renewcommand{\spanishlisttablename}{Lista de tablas}

% Comando que se encarga de componer la portada del TFG
\newcommand{\Portada}{ %
    \thispagestyle{empty}
    \ULCornerWallPaper{1}{imagenes/portada.pdf}
    \begin{textblock*}{14cm}(5cm,13cm) 
        \centering
        {\fontsize{32}{38}\selectfont \textbf{\TituloTFG}}
    \end{textblock*}
    \begin{textblock*}{10cm}(8cm,20.6cm)
      {\fontsize{16}{19}\selectfont \Estudiante}
    \end{textblock*}
    \begin{textblock*}{10cm}(7.2cm,21.9cm)
      {\fontsize{16}{19}\selectfont \TutorA\par\TutorB}
    \end{textblock*}
    ~
    \clearpage
    \ClearWallPaper
    \thispagestyle{empty}
    \definecolor{flashwhite}{rgb}{0.95, 0.95, 0.96}
    \cleardoublepage

    \thispagestyle{empty}
    \begin{figure}
        \centering
        \includegraphics[width=.4\textwidth]{imagenes/uja.jpg}
    \end{figure}
    
    \vspace*{4em}
    
    D./Dª \TutorA \ifdefempty{\TutorB}{}{~y D./Dª \TutorB}, tutor(es) del Trabajo Fin de Grado titulado: \textbf{\TituloTFG}, que presenta \Estudiante, autoriza(n) su presentación para defensa y evaluación en la Escuela Politécnica Superior de Jaén.
    
    \vspace*{2em}
    \begin{center}
        Jaén, \Fecha
    \end{center}
    
    \vspace{2em}
    \hspace{2cm} El estudiante \hspace{6cm} \ifdefempty{\TutorB}{El tutor}{Los tutores}
    
    \vspace{5cm}
    \begin{tabular}{p{6cm}p{.5cm} \ifdefempty{\TutorB}{p{7cm}}{p{4cm}p{4cm}}}
      \Estudiante & & \TutorA \ifdefempty{\TutorB}{}{& \TutorB} \\    
    \end{tabular}
    \clearpage\thispagestyle{empty}
    \onehalfspacing  % Fijamos el interlineado   
}
