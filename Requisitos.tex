% !TeX spellcheck = es_ES

% Cada capítulo de la memoria de TFG comienza con \chapter{TÍTULO DEL CAPÍTULO}, tal y como requiere la normativa de la EPSJ
\chapter{REQUISITOS}  

En este capítulo vamos a presentar los requisitos funcionales que se han llevado a cabo en el desarrollo del trabajo. Para ello vamos a presentar una visión de los mismos de forma detallada.

\section{Requisitos Funcionales}

\begin{itemize}
    \item \textbf{Introducción de la consulta:}
    \begin{itemize}
        \item \textbf{Autor:} Rubén Higueras Gutiérrez
        \item \textbf{Fuentes:} Usuario
        \item \textbf{Descripción:} El sistema deberá permitir la escritura de la consulta para que el usuario pueda comenzar con su búsqueda.
    \end{itemize}
    \item \textbf{Obtención de los datos relevantes:}
    \begin{itemize}
        \item \textbf{Autor:} Rubén Higueras Gutiérrez
        \item \textbf{Fuentes:} Usuario
        \item \textbf{Descripción:} El sistema deberá obtener los datos más relevantes del conjunto dada la consulta para que se añadan a la misma.
    \end{itemize}
    \item \textbf{Introducción de la consulta y los datos relevantes en el LLM:}
    \begin{itemize}
        \item \textbf{Autor:} Rubén Higueras Gutiérrez
        \item \textbf{Fuentes:} Usuario
        \item \textbf{Descripción:} El sistema deberá introducir la consulta y el conjunto de datos relevantes a la misma para que el LLM haga uso de ellos.
    \end{itemize}
    \item \textbf{Obtención del resultado del LLM:}
    \begin{itemize}
        \item \textbf{Autor:} Rubén Higueras Gutiérrez
        \item \textbf{Fuentes:} Usuario
        \item \textbf{Descripción:} El sistema deberá obtener el resultado del LLM en un formato visualizable para que el usuario pueda analizar los datos de su consulta y quedar satisfecho.
    \end{itemize}
    \item \textbf{Actualización de los datos:}
    \begin{itemize}
        \item \textbf{Autor:} Rubén Higueras Gutiérrez
        \item \textbf{Fuentes:} Usuario
        \item \textbf{Descripción:} El sistema deberá actualizar iterativamente los datos con los que se realimentará la consulta para que el usuario reciba información actual en la salida del LLM.
    \end{itemize}
    \item \textbf{Conversión de los datos a un formato visible para el sistema:}
    \begin{itemize}
        \item \textbf{Autor:} Rubén Higueras Gutiérrez
        \item \textbf{Fuentes:} Usuario
        \item \textbf{Descripción:} El sistema deberá convertir los datos que están siendo recibidos a un formato que sea adecuado para que el LLM pueda utilizarlos a la hora de generar una respuesta.
    \end{itemize}
\end{itemize}





